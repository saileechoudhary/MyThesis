\chapter{Malware and Malware Detection\label{chap:blah}}


\section{Malware}

As mentioned in Chapter 1, malware is malicious piece of code. Malware can be subdivided into many different categories. There is a considerable overlap between the various types of malware. Malware includes all families of viruses, worms, backdoor or trapdoor, Trojans, , rabbit, spyware, adware, etc. There are various reasons for which malware can be written. Some may write it as pranks, others with intent of affecting or stealing personal, confidential or business information.    

Let us summarize about different malware families. A virus is malware that relies on someone or something else to propagate from one system to another [3].  A worm is like a virus except that it propagates by itself without the need for outside assistance. This definition implies that a worm uses a network to spread its infection. A trojan horse, or trojan, is software that appears to be one thing but has some unexpected functionality. For example, an innocent-looking game could do something malicious while the victim is playing. A trapdoor or backdoor allows unauthorized access to a system. A rabbit is a malicious program that exhausts system resources. Rabbits could be implemented using viruses, worms, or other means. Spyware is a type of malware that monitors keystrokes, steals data or files, or performs some similar function [4].

We will narrow our discussion to viruses. 

\subsection{Virus}
As discussed above, virus relies on external entities to propagate itself.For example, an email virus attaches itself to an email that is sent from one user to another. Until recently, viruses were the most popular form of malware [3]. The necessary characteristics of a virus are [1]:
\begin{itemize}
\item It should replicate itself.
\item It requires a host program that can act as a carrier.
\item It is activated by some external action.
\item Its replication is limited to (virtual) system.
\end{itemize}

In the side of malware producers, one of the most important issues is to prolong the lifetime of the malware in the wild, as much as possible. There are four main generations in development of viruses :

\subsubsection{Encryption}
For hiding the functionality of the program, the easiest and the first method used by virus writers was encryption. An encrypted virus consists of two parts; the decryptor and the encrypted main body of virus [3]. Virus writers use encryption for the following four reasons, as described in [3]: 
\begin{enumerate}
\item \textbf{To prevent static code analysis:} Static analysis of code involves disassembling the code to examine it for suspicious content such as instructions or blocks of the code. The use of encryption can cover suspicious instructions and so the use of static analysis can fail in encrypted viruses.
\item \textbf{To prolong the process of dissection:} Although encryption makes the analysis of code more difficult, it only adds few extra minutes to the time required for analysis [3]. 
\item \textbf{To prevent tampering:} The process of modification or creation of new variants of virus becomes complex if the virus is encrypted.
\item \textbf{To evade detection:} The early encrypted viruses used an identical decryptor for all infected files, making it easier for detection [3]. However, more sophisticated viruses use self-modifying encryption which makes detection based on the decryptor impossible.
\end{enumerate}

One of the first viruses to use encryption was the DOS virus Cascade [6]. The encryption method of cascade consists of XOR-ing each byte twice with variable values, one of which depends on length of the program. Although such encryption is considered cryptographically weak, early antivirus programs could detect them only by detecting the decryptor using search strings. The decryptor of the Cascade virus is as follows [6]:

\begin{verbatim}
ea      si, Start   ; position to decrypt (dynamically set)
mov     sp, 0682    ; length of encrypted body (1666 bytes)

Decrypt:
xor     [si],si     ; decryption key/counter 1
xor     [si],sp     ; decryption key/counter 2
inc     si          ; increment one counter
dec     sp          ; decrement the other
jnz     Decrypt     ; loop until all bytes are decrypted

Start: 	            ; Encrypted/Decrypted Virus Body
\end{verbatim}

\subsubsection{Oligomorphism}
Oligomorphic viruses are like encrypted viruses, but they differ in that, each new generation changes their decryptor [6]. A simple technique is to carry multiple decryptors instead of one. 

Win95/Memorial could build 96 different decryptor patterns, thus making detection based on the decryptor alone practically impossible. Memorial was the first Windows 95 virus with oligomorphic properties and was the first step towards Windows 95 polymorphism. The code below is an example decryptor of the Memorial virus published in [6]:

\begin{verbatim}
mov     ebp,00405000h       ; select base
mov     ecx,0550h           ; this many bytes
lea     esi,[ebp+0000002E]  ; offset of "Start"
add     ecx,[ebp+00000029]  ; plus this many bytes
mov     al,[ebp+0000002D]   ; pick the first key

Decrypt:
nop                  ; junk
nop                  ; junk
xor     [esi],al     ; decrypt a byte
inc     esi          ; next byte
nop                  ; junk
inc     al           ; slide the key
dec     ecx          ; are there any more bytes to decrypt?
jnz     Decrypt      ; until all bytes are decrypted
jmp     Start        ; decryption done, execute body

; Data area

Start:
; encrypted/decrypted virus body

\end{verbatim}

The order of the instructions can be changed to some extent and the decryptor can use different instructions for looping [6]. A slightly different version of decryptor for the same virus is as follows [6] :
\begin{verbatim}
mov     ecx,0550h           ; this many bytes
mov     ebp,013BC000h       ; select base
lea     esi,[ebp+0000002E]  ; offset of "Start"
add     ecx,[ebp+00000029]  ; plus this many bytes
mov     al,[ebp+0000002D]   ; pick the first key

Decrypt:
nop                         ; junk
nop                         ; junk
xor     [esi],al            ; decrypt a byte
inc     esi                 ; next byte
nop                         ; junk
inc     al                  ; slide the key
loop    Decrypt             ; until all bytes are decrypted
jmp     Start               ; Decryption done, execute body
					
; Data area

Start:
; Encrypted/decrypted virus body
\end{verbatim}

\subsubsection{Polymorphism}
Similar to an encrypted virus, a polymorphic virus includes a virus body and a decryption routine. The decryption routine gains control of the computer, and then decrypts the virus body. However, a polymorphic virus adds to these two components a third \textemdash a mutation engine that generates randomized decryption routines that change each time a virus infects a new program [10]. The mutation engine as well as the virus body are encrypted. When an infected program is executed by the user, the decryptor takes control of the computer and then decrypts both the virus body and the mutation engine. 

The first known polymorphic virus written was 1260 [6]. The virus inserts junk instructions in its decryptor and uses two sliding keys to decrypt its body. The junk instructions have no function other than altering the appearance of the code. Simple search strings could not be extracted from the code. Even though the decryptor is simple, it can become shorter or longer based on the number of junk instructions inserted. Example of one of the instance taken from [6] is as follows :

\begin{verbatim}
; Group 1  Prolog Instructions
inc     si        ; optional, variable junk
mov     ax,0E9B   ; set key 1
clc               ; optional, variable junk
mov     di,012A   ; offset of Start
nop               ; optional, variable junk
mov     cx,0571   ; this many bytes - key 2
\end{verbatim}

\begin{verbatim}
; Group 2  Decryption Instructions
Decrypt:
xor     [di],cx   ; decrypt first word with key 2
sub     bx,dx     ; optional, variable junk
xor     bx,cx     ; optional, variable junk
sub     bx,ax     ; optional, variable junk
sub     bx,cx     ; optional, variable junk
nop               ; non-optional junk
xor     dx,cx     ; optional, variable junk
xor     [di],ax   ; decrypt first word with key 1
\end{verbatim}

\begin{verbatim}
; Group 3  Decryption Instructions
inc     di        ; next byte
nop               ; non-optional junk
clc               ; optional, variable junk
inc     ax        ; slide key 1
; loop
loop    Decrypt   ; until all bytes are decrypted  slide key 2
; random padding up to 39 bytes
\end{verbatim}

Each group contains five non-repeated junk instructions. Two nop instructions always appear. 1260 is an successful polymorphic engine that generates a high range of decryptors.

\subsubsection{Metamorphism}
Peter Szor quoted in [6], the shortest definition of the metamorphic virus, which was defined by Igor Muttik, is "Metamorphics are body-polymorphics."  Metamorphic viruses do not use encryption, and hence the decryptor is not needed. It is similar to polymorphic viruses in a way that it uses a mutation engine for replicating itself. Each new copy may have different structure, code sequence, size and syntactic properties, but the behavior of the virus does not change [11]. Metamorphic computer viruses have the ability to change their shape by themselves from one form to another, but they usually avoid generating instances that are very close to their parent shape [6].

There are various techniques that can be used to create metamorphic viruses like dead code insertion, subroutine permutation, etc. We will focus more on metamorphic viruses and its detection evasion techniques in the Chapter 3.

\section{Malware Detection}
There are three general approaches that can be used to detect viruses [4]. The first one is signature detection, which counts on finding a signature or pattern that is present in malware. The second detection technique is change detection. It detects the files that have changed unexpectedly, as they might indicate infection. The third and the last kind of detection is anomaly detection. The aim of this detection technique is to discover any unusual behavior in the files. We will have a look at each of these detection technique in detail.

\subsection{Signature Detection}
Signature based virus detection is the most common technique employed in antivirus software for identifying viruses. The signature of the file is computed according to the contents of the file and this signature is compared with the database of the signatures that are already present in the antivirus software. If a signature match is found, the file is considered as infected and needs to be repaired or deleted from the system.

For example, according to [6], the signature used for the W32/Beast virus is \textbf{83EB 0274 EBOE 740A 81EB 0301 0000}. After searching in the system for this signature in all files, we cannot be sure that it is infected, since a benign file can also contain this signature. If the bits in searched files were random, the chance of such a false match would be 1/2112. However, computer software and data is far from random, so there is probably some realistic chance of a false match. This means that if a matching signature is found, further testing may be required to be certain that it actually represents the W32/Beast virus [4].

Advantage of Signature based virus detection [4]:  
\begin{itemize}
\item This type of malware detection technique is effective for known malware and whose signature can be easily extracted.
\item It requires less work from users and administrator end as only signature files need to be updated.
\end{itemize}

Disadvantage of Signature based virus detection [4]:  
\begin{itemize}
\item The signature files can become too lengthy and thus could result in slower scanning.
\item Signature detection works for only known signatures, even a slight variation can cause a miss in detection of an infected file.
\end{itemize}

\subsection{Change Detection}
If we detect a change somewhere on the system, it may be an indication of infection. This is known as change detection. One method of implementing change detection is by using hash functions. For using this method, we first calculate the hashes of the files and store them on the system securely. At regular intervals, we re-calculate the hash values of the files and compare them with these stored hashes. If any changes have been found in hash values, we can check them for infection.

Advantages of change detection [4]:
\begin{itemize}
\item There are no false negatives, as changes in hash values will always be detected.
\item Previously unknown malware can be detected.
\end{itemize}

Disadvantages of change detection [4]:
\begin{itemize}
\item Many false positives can be encountered as many files on the system often change. This may be a burden for the users and administrators.
\end{itemize}

\subsection{Anomaly Detection}
Anomaly detection is aimed at finding any unusual or virus-like or other potentially malicious activity or behavior [4]. The elementary challenge in anomaly detection is to decide what is normal and what is abnormal. Another challenge is that the definition of normal can change frequently as the requirements of the system change. So, this detection technique can give false positives. 

Advantages of anomaly detection system [4]:
\begin{itemize}
\item A patient attacker may be able to make an anomaly appear to be normal.
\item Previously unknown malware can be detected.
\end{itemize}

Disadvantages of anomaly detection system [4]:
\begin{itemize}
\item It cannot work as a standalone system and hence is often combined with signature detection. 
\end{itemize}

Apart from these general approaches that are used for virus detection, various machine learning techniques such as Hidden Markov Models (HMM) can be used for virus detection as mention in Chapter 1. Chapter 5 provides detailed understanding of HMM and its working. 