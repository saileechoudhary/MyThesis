\chapter{Introduction}

In today's world, we prefer to get things handily with no effort. The majority of transactions, forms and other sensitive information is accessed from computers. It is pivotal to treat information security as a concern of supreme importance. Today, computer systems are under attack from a multitude of sources. These range from malicious code, such as viruses and worms, to human threats, such as hackers and phone freaks.  Viruses have "evolved" over the years due to efforts by their authors to make the code more difficult to detect, disassemble, and eradicate [1].

Malware is short for malicious software. It is code or software that is specifically designed to damage, disrupt, steal, or in general inflict some other bad or illegitimate action on data, hosts, or networks. [2] Viruses and worms are classes of malicious code. They self-replicate and spread copies of themselves. To be classified as a virus or worm, malware must have the ability to propagate itself. 

Metamorphic malware is rewritten with each iteration, so that each succeeding version of the code is different every time. Metamorphic viruses use different code obfuscation techniques to change the structure of the code. These techniques include subroutine permutation, dead code insertion, code reordering through jumps, equivalent instruction substitution, and rearrangement of instruction order (transposition). There are certain bytecode manipulation libraries that can be used to obfuscate code using dead code insertion, instruction substitution, etc. Some of these libraries are ASM, Javassist, BCEL, CGLib. These libraries allow us to create classes on the fly or modify existing classes directly in the binary form.  

The anti-detection techniques are also getting very sophisticated. The code changes makes it difficult for signature-based antivirus software programs to detect metamorphic viruses. The antivirus programs do not recognize that different iterations are the same malicious program. To be specific, paper [28] provides an evidence that metamorphic viruses can evade signature-based detection, if it has been built using specified guidelines. [26] Hidden Markov models (HMMs) have been reliable for statistical pattern analysis. These models use machine learning techniques, and probability for each file can be determined against the trained model, to check whether it belongs to same virus family as the training set. 

The aim of this project is to develop a metamorphic engine using code obfuscation techniques which include  dead code insertion, subroutine permutation and instruction substitution. The obfuscation is performed on Java class files using ASM bytecode manipulation library for achieving obfuscation. As the first step for detection, signature-based detection is performed using anti-virus detectors. For more efficient detection, HMM models are used. Both threshold as well as dueling HMM are used for the experiments.
 
This paper is arranged as follows :
\begin{itemize}
\item Chapter 2 describes malware and its classification. It also describes the different techniques used in detecting malware. 
\item Chapter 3 gives a detailed look at metamorphic malware and, different code obfuscation techniques. 
\item Chapter 4 provide accurate details about different bytecode manipulation libraries.
\item Chapter 5 introduces Hidden Markov Models in-depth. 
\item Chapter 6 focuses on a detailed look at the design and implementation of this project.
\item Chapter 7 contains the experiments and the results of the experiments.
\item Chapter 8 draws the conclusion based on our findings and explains the possible future work.

\end{itemize}
