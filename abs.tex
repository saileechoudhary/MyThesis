\begin{abstract}
`Malware refers to software programs designed to damage or do other unwanted actions on a computer system. Metamorphic malware is a category of malicious software programs that has the ability to change its code as it propagates. A Hidden Markov model (HMM) is a statistical Markov model in which the system being modeled is assumed to be a Markov process with unobserved states. A HMM is based on use of statistics to detect patterns, and hence in metamorphic virus detection. Previous work has been done in order to create morphing engines using LLVM-bytecode format, which have been successfully detected by HMM.

   This project includes creation of morphing engine for Java bytecode, using different code obfuscation techniques. The next aspect is to focus on detection techniques, specifically signature-based and HMM for validation of the created engine. The paper also proves that if set of rules have been followed while creation of metamorphic engine, signature-based detection cannot detect the presence of morphing. Also, the results presented prove that HMM can successfully detect the morphing. 
\end{abstract}
